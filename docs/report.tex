%                        
% Шаблон отчета о научно-исследовательской работе (практике) студентов
%
% Дата последнего изменения: 11.10.2020
%

%%%%%%%%%%%%%%%%%%%%%%%%%%%%%%%%%%%%%%%%%%%%%%%%%%%%%%%%%%%%%%%%%%%%%%%%%%%%%%%%
%                         
% Преамбула документа содержит начальные настройки форматирования,
% редактируйте с осторожностью

\documentclass[a4paper,12pt]{article}

\usepackage{ucs}
\usepackage[utf8x]{inputenc}
\usepackage[russian]{babel}
\usepackage{hyphenat}
\usepackage{graphicx}
\usepackage{listings}
\usepackage{xcolor}
% \usepackage{paratype}

% Включаем библиографию в содержание без номера
\usepackage[nottoc,notlot,notlof]{tocbibind}

% Формат пункта библиографического перечня
\makeatletter
\renewcommand\@biblabel[1]{#1.}
\makeatother

% Позиция подписи
\newcommand{\myrule}[1]{\rule{#1}{0.4pt}}
\newcommand{\sign}[2][~]{{\small\myrule{#2}\\[-0.7em]\makebox[#2]{\it #1}}}

% Поля страницы
\usepackage[top=20mm, left=30mm, right=10mm, bottom=20mm, nohead]{geometry}

% Красная строка в том числе для первого абзаца раздела
\usepackage{indentfirst}

% Межстрочный интервал
\renewcommand{\baselinestretch}{1.5}

% Согласно ГОСТу в заголовках таблиц, листинго кода, рисунков
% в качестве разделителя номера и текста заголовка используется тире 
\usepackage[labelsep=endash]{caption} 
\captionsetup[table]{skip=1ex}

% Размер полосы разделителя между столбцами таблицы по умолчанию
\setlength{\tabcolsep}{1em}

% Формат листинга по умолчанию
\lstdefinestyle{mylisting}{%
    basicstyle=\ttfamily,
    columns=fullflexible,
    keepspaces=true,
    upquote=true,
    commentstyle=\normalshape,
    keywordstyle=\bfseries,
    showstringspaces=false,
    captionpos=t,
    belowcaptionskip=1.5ex,
    frame=lines
}

%
% Конец преамбулы
%
%%%%%%%%%%%%%%%%%%%%%%%%%%%%%%%%%%%%%%%%%%%%%%%%%%%%%%%%%%%%%%%%%%%%%%%%%%%%%%%%


\begin{document}

%%%%%%%%%%%%%%%%%%%%%%%%%%%%%%%%%%%%%%%%%%%%%%%%%%%%%%%%%%%%%%%%%%%%%%%%%%%%%%%%
%                         
% Начало титульного листа 
%
% отредактируйте в соответствии с указаниями в комментариях

% Титульный лист не нумеруется
\thispagestyle{empty}

% Образовательная организация и подразделение
%
% Раскомментируйте (уберите знак процента в начале строки) в наименовании
% кафдеры руководителя
\begin{center}
  \renewcommand{\baselinestretch}{1}
  {\large
    {\sc
      Петрозаводский государственный университет\\
      Институт математики и информационных технологий\\
      %\textcolor{red}{Раскомментируйте наименование кафедры руководителя}\\
      Кафедра информатики и математического обеспечения\\
      % Кафедра математического анализа\\
      % Кафедра прикладной математики и кибернетики\\
      % Кафедра теории вероятностей и анализа данных\\
      % Кафедра теории и методики обучения\\ математике и ИКТ в образовании\\
    }
  }
\end{center}

% Направление подготовки студента и профиль обучения
%
% Раскомментируйте (уберите знак процента в начале строки) для одной из строк
% типа направления  - бакалавриат / магистратура и для трёх строк 
% вашего направление подготовки и профиля обучения
\begin{center}
  %\textcolor{red}{Раскомментируйте ваше направление и профиль обучения} 
  % Направление подготовки бакалавриата\\

  % 01.03.01 -- Математика\\
  % Профиль направления подготовки бакалавриата\\
  % Математика в образовании, фундаментальных и прикладных исследованиях\\
  
  % 44.03.05 -- Педагогическое образование (с двумя профилями подготовки)\\
  % Профиль направления подготовки бакалавриата\\
  % Образование в предметных областях (математика и информатика)\\
  
  % 09.03.02 -- Информационные системы и технологии\\
  % Профиль направления подготовки бакалавриата\\
  % Разработка информационных систем\\
  
  09.03.04 -- Программная инженерия\\
  % Профиль направления подготовки бакалавриата\\
  % Системное и прикладное программное обеспечение\\
  
  % Направление подготовки магистратуры\\
  
  % 01.04.02 -- Прикладная математика и информатика\\
  % Профиль направления подготовки магистратуры\\
  % Математическое моделирование и информационно-коммуникационные технологии,
  % Интернет вещей\\
  
  % 09.04.02 -- Информационные системы и  технологии\\
  % Профиль направления подготовки магистратуры\\
  % Управление данными\\
\end{center}

\vfill

% Наименование отчёта и тема работы
%
% Замените отмеченный красным цветом текст на свои данные
\begin{center}
  {\normalsize
    %\textcolor{red}{Раскомментируйте название отчёта для своего направления}
    % Для студентов всех направлений, кроме 09.03.04
    % Отчет о практике по научно-исследовательской работе\\
	Проектная работа по курсу "Основы информатики и программирования"
	% Отчет о научно-исследовательской практике\\
  }
  
  \medskip

  % Замените на название работы
  {\Large
    Приложение по распознаванию лиц
  } \\
  
  % Для промежуточного отчёта (в осеннем семестре оставьте следующую строку,
  % для окончательного отчёта (в весеннем семестре) удалите её
  %(промежуточный)
\end{center}

\medskip

% Блок подписей и оценок
%
% Замените отмеченный красным цветом текст на свои данные
% Девушкам следует применять слова "Выполнила" и "студентка"
\begin{flushright}
  \parbox{11cm}{%
    \renewcommand{\baselinestretch}{1.2}
    \normalsize
	Выполнил:\\
    студент 1 курса группы 22107
    \begin{flushright}
	  К.А. Смирнов \sign[подпись]{4cm}
    \end{flushright}

%    Место прохождения практики:\\
 %   \textcolor{red}{наименование кафедры или предприятия}\\

   % Период прохождения практики:\\
    % Раскомментируйте строку с нужным Вам периодом прохождения.
    % Для 4 к. бакалавриата и 6 к. магистратуры:\\
    % Период прохождения практики: \\ ДД.ММ.ГГ--ДД.ММ.ГГ\\
    % Для остальных курсов:\\
    % Период прохождения практики: \\ ДД.ММ.ГГ--ДД.ММ.ГГ\\

    % Если руководителей два, замените следующую строку на "Руководители" и
    % раскомментируйте и заполните блок с данными второго руководителя
    % Степень и звание руководителя можно уточнить непосредственно у руководителя
    % или на сайте университета (petrsu.ru)

    Руководитель:\\
    % Заполните данными первого руководителя 
    А.В. Бородин, старший преподаватель \\
    \begin{flushright}
      \sign[подпись]{4cm}
    \end{flushright}
    
    % Заполните данными второго руководителя (если применимо)
    % \textcolor{red}{И. О. Фамилия, ученая степень, ученое звание} \\
    % \begin{flushright}
    %   \sign[подпись]{4cm}
    % \end{flushright}
    
    %Итоговая оценка:
   % \begin{flushright}
    %  \sign[оценка]{4cm}
   % \end{flushright}
  }
\end{flushright}

\vfill

\begin{center}
\large
    Петрозаводск --- 2021
\end{center}

%
% Конец титульного листа  
%                         
%%%%%%%%%%%%%%%%%%%%%%%%%%%%%%%%%%%%%%%%%%%%%%%%%%%%%%%%%%%%%%%%%%%%%%%%%%%%%%%%

\newpage

%%%%%%%%%%%%%%%%%%%%%%%%%%%%%%%%%%%%%%%%%%%%%%%%%%%%%%%%%%%%%%%%%%%%%%%%%%%%%%%%
%
% Содержание 

\tableofcontents

% Содержание
% 
%%%%%%%%%%%%%%%%%%%%%%%%%%%%%%%%%%%%%%%%%%%%%%%%%%%%%%%%%%%%%%%%%%%%%%%%%%%%%%%%

\newpage

%%%%%%%%%%%%%%%%%%%%%%%%%%%%%%%%%%%%%%%%%%%%%%%%%%%%%%%%%%%%%%%%%%%%%%%%%%%%%%%%
%                          
% Введение                 

% В введении Вы должны описать предметную область, с которой связана
% Ваша работа, показать её актуальность, определить цель и задачи
% исследования/разработки
%
% Обратите внимание, все фактические сведения, полученные из сторонних
% источников и упомянутые в тексте работы, должны сопровождаться ссылками
% на источники.

\section*{Введение}
\addcontentsline{toc}{section}{Введение}

Цель проекта: Разработать приложение по распознаванию лиц на цифровых изображениях (или в потоке изображений) \\

Задачи:
\begin{enumerate}
\item Ознакомиться с функциями библиотеки OpenCV 
\item Разработать функции по обнаружению и выделению лица на входящем изображениями
\item Реализовать функции обработки действий пользователя
\item Разработать графический интерфейс приложения
\item Реализовать приложения с помощью Qt Widgets
\end{enumerate}

В настоящее время все большую популярность набирают сферы деятельности, в которых задействуется использование нейросетей. Примером может послужить компьютерное зрение, которое все больше становится частью жизни современного общества. Компьютерное зрение имеет широкий спектр применения, например, в концептах беспилотных автомобилей Tesla используется именно эта технология. Цель этого проекта: разработать подобное приложение, основанное на использовании нейросети для распознавания лиц.

% Конец введения
%
%%%%%%%%%%%%%%%%%%%%%%%%%%%%%%%%%%%%%%%%%%%%%%%%%%%%%%%%%%%%%%%%%%%%%%%%%%%%%%%%

%%%%%%%%%%%%%%%%%%%%%%%%%%%%%%%%%%%%%%%%%%%%%%%%%%%%%%%%%%%%%%%%%%%%%%%%%%%%%%%%
%
% Основная часть текста работы
\newpage

\section{Требования к приложению}

\begin{itemize}
\item Распознавание лиц "в прямом эфире" в потоке изображений с веб-камеры разрешением от 320х240
\item Распознавание лиц на выбранных пользователем изображениях формата .png и .jpg
\item Распознавание лиц на выбранных пользователем видеозаписях формата .mp4
\end{itemize}

\newpage

\section{Разработка необходимых функций}

Для распознавания лиц необходимо обучить каскад Хаара, чтобы нейросеть понимала, что такое "лицо" и как его идентифицировать. Однако, библиотека OpenCV содержит уже обученные каскады, которыми было решено и воспользоваться, так как они основаны на большой обучающей выборке и показывают хорошую степень распознавания. \\

\begin{itemize}
\item showVideo - функция, которая вызывается для событий, требующих обработки потока изображений: анализ веб-камеры и анализ видеофайла. Особенностью этой функции является перехват каждого полученного кадра, отправка его функции-обработчику по обнаружению наличия лица, а также вывод итогового изображения на экран.
\item highlightParts - функция по обнаружению и выделению лиц на входном изображении с использованием каскада Хаара. Для реализации используются функции библиотеки OpenCV, такие как .detectMultiScale, которая определяет наличие лица, Rectangle, обводящая искомое лицо в прямоугольник и Circle, которая выделяет кругами глаза.
\end{itemize}

\newpage

\section*{Заключение}
\addcontentsline{toc}{section}{Заключение}

В результате было разработано приложение с простеньким графическим интерфейсом, которое соответствует требованиям, а также работает довольно быстро. Пользователь может проанализировать практически любой графический файл на наличие на нем лица. Отличительной особенностью является тот факт, что алгоритмы обработки событий (какой анализ выбран) и функции обработчики никак не связаны с искомым объектом, то есть всего лишь заменив файла каскада Хаара на желаемый (например, вместо лица - автомобильный номер) можно получить приложение по распознаванию уже других объектов.

\end{document}
